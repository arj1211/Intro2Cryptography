\chapter{A Math Topic}

\section{Introduction to Mathematical Concepts}

In this chapter, we explore several mathematical topics, including algebra, calculus, and linear algebra. The following examples demonstrate various mathematical expressions and environments in \LaTeX.

\subsection{Basic Equations}

Consider the quadratic formula:
\[
x = \frac{-b \pm \sqrt{b^2 - 4ac}}{2a}.
\]
This formula provides the solutions to the quadratic equation
\[
ax^2 + bx + c = 0.
\]

\subsection{Systems of Equations}

An example of a linear system is:
\begin{align*}
x + 2y &= 3, \\
3x - y &= 7.
\end{align*}
Solving these equations simultaneously yields the values of \(x\) and \(y\).

\subsection{Integrals and Summations}

A classic integral:
\[
\int_0^\infty e^{-x} \, dx = 1.
\]
And a geometric series summation:
\[
\sum_{n=0}^{\infty} ar^n = \frac{a}{1-r}, \quad \text{for } |r| < 1.
\]

\subsection{Matrices and Determinants}

A typical \(2 \times 2\) matrix is expressed as:
\[
A = \begin{pmatrix} a & b \\ c & d \end{pmatrix},
\]
with its determinant given by:
\[
\det(A) = ad - bc.
\]

\subsection{Theorem and Proof Example}

% Define a new theorem environment named 'pythm' for our Pythagorean Theorem.
\newtheorem{pythm}{Pythagorean Theorem}

\begin{pythm}
For any right-angled triangle with legs of lengths \(a\) and \(b\), and hypotenuse \(c\), the following holds:
\[
a^2 + b^2 = c^2.
\]
\end{pythm}

\begin{proof}
One common proof involves constructing squares on each side of the triangle and comparing their areas. By rearranging the areas of the two smaller squares, one can show that their combined area equals the area of the square on the hypotenuse.
\end{proof}

\subsection{Derivatives and Exponential Functions}

The derivative of the exponential function is:
\[
\frac{d}{dx} e^x = e^x.
\]
And for the natural logarithm:
\[
\frac{d}{dx}\ln(x) = \frac{1}{x}.
\]

\subsection{Complex Numbers}

A complex number is written as:
\[
z = x + iy, \quad \text{where } i^2 = -1.
\]
The modulus of \(z\) is defined by:
\[
|z| = \sqrt{x^2 + y^2}.
\]

\subsection{Additional Concepts}

We can also represent series, limits, and more advanced topics. For example, the limit definition of the exponential function is:
\[
e = \lim_{n\to\infty}\left(1 + \frac{1}{n}\right)^n.
\]

This chapter has provided a sample of diverse mathematical content. Feel free to expand on any of these sections as you continue building your notes!
