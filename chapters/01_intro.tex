\chapter{Introduction}
\section{Divisibility, GCD}
\begin{definition}[Divisibility]
  \begin{gather*}
    \forall a,b \in \Z, \ b \ne 0,	\\
    \exists c \in \Z : \ifthen{a=bc}{b \ \textbf{divides} \ a}
  \end{gather*}
  also written \[\divides{b}{a}\]
  and if $b$ \underline{doesn't} divide $a$, \[\ndivides{b}{a}\]
\end{definition}
\subsection{Properties of Divisibility}
\begin{enumerate}
  \item
        $\ifthen{a|b \wedge b|c}{a|c}$	\\
        $\ifthen{a|b}{b=ax_1, \ \exists x_1 \in \Z}$	\\
        $\ifthen{b|c}{c=bx_2, \ \exists x_2 \in \Z}$	\\
        $\Rightarrow \ifthen{c=ax_1x_2}{c=ax_3, \ \exists x_3 \in \Z}$	\\
        $\Rightarrow a|c \qed$
  \item
        $\ifthen{a|b \wedge b|a}{a=\pm b}$	\\
        $\ifthen{b=ax_1, \ a=bx_2}{x_1x_2=1}$	\\
        $\Rightarrow \ifthen{x_1=x_2=\pm 1}{a=\pm b} \qed$
  \item
        $\ifthen{a|b \wedge a|c}{a|(b+c) \wedge a|(b-c)}$  \\
        $\ifthen{b=ax_1, \ c=ax_2}{
            \begin{aligned}
              b+c & = a(x_1+x_2) = \ifthen{ax_3}{a|(b+c)}      \\
              b-c & = a(x_1-x_2) = \ifthen{ax_3}{a|(b-c)} \qed
            \end{aligned}
          }$
\end{enumerate}
\subsection{Greatest Common Divisors}
\begin{definition}[Common Divisor]
  $\ifthen{d \in \Z^+ : d|a \wedge d|b}{d \ \text{is a \textbf{common divisor} of} \ a,b \in \Z}$
\end{definition}
\begin{definition}[Greatest Common Divisor]
  $\ifthen{d = \max \{\forall d : d \ \text{is a \textbf{common divisor} of} \ a,b \in \Z\} \\}{d \ \text{is the \textbf{greatest common divisor} of} \ a,b \in \Z, \text{denoted} \ d=gcd(a,b)}$
\end{definition}
\begin{definition}[Division with Remainder]
  $\ifthen{a=bq+r, \ 0 \leq r < b, \ \forall a,b \in \Z^+ \\}{\text{$a$ divided by $b$ has \textbf{quotient} $q$ with \textbf{remainder} $r$}}$
\end{definition}
Lets say we want $gcd(a,b)$, first divide $a$ by $b$ $\Rightarrow$ $a=bq+r, \ 0 \leq r < b$ \\
$\begin{aligned}
    \ifthen{\exists d & : d|a \wedge d|b}{d|r}                                      \\
                      & \ifthen{a=dx_1, \ b=dx_2}{dx_1=dx_2q+r}                     \\
                      & \Rightarrow r = dx_1-dx_2q=d(x_1-x_2q), \  x_1,x_2,q \in \Z \\
                      & \Rightarrow \ifthen{r=dx_3}{d|r} \qed
  \end{aligned}$\\
$\begin{aligned}
    \ifthen{\exists e & : e|b \wedge e|r}{e|a}                                     \\
                      & \ifthen{b=ex_1, \ r=ex_2}{\ifthen{a=ex_2q+ex_2}{e|a}} \qed
  \end{aligned}$\\
$\begin{aligned}
     & (\ifthen{d|a \wedge d|b}{d|r}) \wedge (\ifthen{e|b \wedge e|r}{e|a})         \\
     & \Rightarrow \text{all common divisors of $a,b$ are common divisors of $b,r$} \\
     & \Rightarrow gcd(a,b) = gcd(b,r)
  \end{aligned}$\\
\\
So with
$$a=bq+r, \ 0 \leq r < b$$ \\
repeating the step with $b,r$ \\
$$b=rq^\prime+r^\prime, \ 0 \leq r^\prime < r$$ \\
$\Rightarrow gcd(b,r) = gcd(r, r^\prime)$ \\
\\
With each iteration, the \underline{remainder term decreases} till it hits zero \\
then\\
\begin{align*}
  gcd(s,0) & = s        \\
           & = gcd(a,b)
\end{align*}\\
\begin{example}
  Find $gcd(2024,748)$
  \begin{align*}
    2024         & = \mathbf{748} \cdot 2 + 528 \\
    \mathbf{748} & = 528 \cdot 1 + \mathbf{220} \\
    528          & = \mathbf{220} \cdot 2 + 88  \\
    220          & = 88 \cdot 2 + 44            \\
    88           & = 44 \cdot 2 + 0             \\
                 & \Rightarrow gcd(2024,748)=44
  \end{align*}
\end{example}
\subsection{Euclidean Algorithm}
\begin{definition}[Euclidean Algorithm]
  The \textbf{Euclidean Algorithm} is as follows
  \begin{algorithmic}
    \State $r_0 \gets a, r_1 \gets b$
    \State $i \gets 1$
    \While {$r_{i+1} \neq 0$}
    \State $r_{i-1} \gets r_i \cdot q_i + r_{i+1}, \ 0 \leq r_{i+1} < r_i$
    \State $i \gets i + 1$
    \EndWhile \\
    Now $\ifthen{r_{i+1}=0}{r_i=gcd(a,b)}$ \\
    \Return $r_i$
  \end{algorithmic}
\end{definition}
\section{Modular Arithmetic}
\section{Primes, Factorizations, Primitive Roots}
\section{Symmetric, Asymmetric Ciphers}